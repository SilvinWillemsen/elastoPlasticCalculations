\documentclass{article}
\usepackage[utf8]{inputenc}

\title{Elasto-Plastic Calculations}
\author{sil }
\date{March 2019}
\usepackage[left=3cm, right=3cm, top=2cm]{geometry}
\usepackage{natbib}
\usepackage{graphicx}
\usepackage{appendix}
\usepackage{amsmath}
\usepackage{amsfonts}
\usepackage{amssymb}
\newenvironment{rcases}
  {\left.\begin{alignedat}{2}}
  {\end{alignedat}\right\rbrace}
\DeclareMathOperator{\sgn}{sgn}
\begin{document}

\maketitle

\section{Introduction}
\section{Elasto-Plastic}
At the bowing point $l=x_\text{B}$, the FDS for the bowed string looks like:
\begin{equation}
\label{eq:FDS}
\delta_{tt} u_l^n =c^2 \delta_{xx} u_l^n -\kappa^2\delta_{xxxx} u_l^n - 2\sigma_0\delta_{t\cdot} u_l^n
+ 2\sigma_1\delta_{t-}\delta_{xx}u_l^n - f(v_\text{rel}, z),
\end{equation}
where we need to iteratively solve for two unknown variables: the relative velocity between the bow and the string $v_\text{rel}$ and the mean bristle displacement $z$ of the bow.
The excitation (or bowing) function is defined as
\begin{equation}
    f(v_\text{rel}, z) = s_0z + s_1\dot z + s_2v_\text{rel}
\end{equation}
where 
\begin{equation}\label{eq:zdot}
    \dot z(v_\text{rel}, z) = v_\text{rel}\bigg[1-\alpha(v_\text{rel}, z)\frac{z}{z_\text{ss}(v_\text{rel})}\bigg]
\end{equation}
with adhesion map
\begin{equation}
\alpha(v_\text{rel}, z) 
    \begin{cases}
    \begin{rcases}
        &0 & |z| < z_\text{ba}\\
       &\alpha_\text{m}(v_\text{rel},z)&\quad \  z_\text{ba}<|z|<z_\text{ss}(v_\text{rel})\\
        &1 &|z|>z_\text{ss}(v_\text{rel})
        \end{rcases}\text{if}\  \sgn(v_\text{rel})=\sgn(z)\\
        0\qquad \qquad \qquad \qquad \qquad \qquad \qquad\ \ \; \text{if}\  \sgn(v_\text{rel})\neq\sgn(z),
    \end{cases}
\end{equation}
the transition between the elastic and plastic behaviour
\begin{equation}
    \alpha_\text{m} = \frac{1}{2}\bigg[1+\sin\bigg(\pi\frac{z-\frac{1}{2}(z_\text{ss}(v_\text{rel})+z_\text{ba})}{z_\text{ss}(v_\text{rel})-z_\text{ba}}\bigg)\bigg],
\end{equation}
and the steady-state function
\begin{equation}
    z_\text{ss} = \frac{\sgn(v_\text{rel})}{s_0}\Big[f_\text{c}+(f_\text{s}-f_\text{c})e^{-(v_\text{rel}/v_\text{s})^2}\Big].
\end{equation}
We can solve \eqref{eq:FDS} this we can use the following identities
\begin{equation}
    \delta_{tt}u_l^n = \frac{2}{k}\big(\delta_{t\cdot}u_l^n-\delta_{t-}u_l^n\big) \quad \text{ and } \quad \delta_{t\cdot}u_l^n = v_\text{rel} + v_\text{B},
\end{equation}
resulting in 

\begin{equation}
\label{eq:stiffStringFDS}
\frac{2}{k}v_\text{rel} + \frac{2}{k}v_\text{B} - \delta_{t-}u_l^n =c^2 \delta_{xx} u_l^n -\kappa^2\delta_{xxxx} u_l^n - 2\sigma_0v_\text{rel} - 2\sigma_0v_\text{B}
+ 2\sigma_1\delta_{t-}\delta_{xx}u_l^n - f(v_\text{rel}, z).
\end{equation}
This can be rewritten to
\begin{align}\label{eq:newtonFunction}
    &s_0z+s_1\dot z+s_2v_\text{rel} + \frac{2}{k}v_\text{rel} + 2\sigma_0v_\text{rel} + b = 0 \quad \text{where} \\
    & b = \frac{2}{k}v_\text{B}-\frac{2}{k}\delta_{t-}u_l^n - c^2 \delta_{xx} u_l^n +\kappa^2\delta_{xxxx} u_l^n + 2\sigma_0v_\text{B}
- 2\sigma_1\delta_{t-}\delta_{xx}u_l^n,
\end{align}
where $b$ can be pre-computed.

Newton's method (or Newton-Raphson) is defined as
\begin{equation}
    x^{i+1} = x^{i} - \frac{g(x^i)}{g'(x^i)}
\end{equation}
where $g(x)$ is an arbitrary function dependent on to-be-calculated variable $x$.
As we need to find the roots of \eqref{eq:zdot}, $g$ with $x=\dot z$ is defined as
\begin{equation}
   g(\dot z) = v\bigg[1-\alpha(v, z)\frac{z}{z_\text{ss}(v)}\bigg] -  \dot z(v,z) = 0,
\end{equation}
and
\begin{equation}
    \dot z^{i+1} = \dot z^i - \frac{g(\dot z^i)}{g'(\dot z^i)}
\end{equation}
where
\begin{equation}
    g'(\dot z^i) = -0.2295\frac{dg(\dot z^i)}{dv} + \frac{k}{2}\frac{dg(\dot z^i)}{dz} - 1.
\end{equation}
In the iteration, we use the newly calculated value for $\dot z$ and the value of $z$ at the previous time step to calculate an estimate of $z$ using the trapezoidal rule
\begin{equation}
    z^n = z^{n-1} + \frac{k}{2}(\dot z^{i+1} + \dot z^{n-1}).
\end{equation}
Inserting this into \eqref{eq:newtonFunction} we can caluclate $v_\text{rel}$ using
\begin{equation}
    v_\text{rel} = \frac{-s_0z-s_1\dot z-b}{s_2 + \frac{2}{k} + 2\sigma_0}.
\end{equation}
\end{document}
