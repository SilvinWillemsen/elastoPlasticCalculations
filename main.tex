\documentclass{article}
\usepackage[utf8]{inputenc}

\title{Elasto-Plastic Calculations}
\author{Silvin Willemsen}
\date{June 2019}
\usepackage[left=3cm, right=3cm, top=2cm]{geometry}
\usepackage{natbib}
\usepackage{graphicx}
\usepackage{appendix}
\usepackage{amsmath}
\usepackage{amsfonts}
\usepackage{amssymb}
\usepackage[]{algorithm2e}
\usepackage{textcomp}
\newenvironment{rcases}
  {\left.\begin{alignedat}{2}}
  {\end{alignedat}\right\rbrace}
\DeclareMathOperator{\sgn}{sgn}
\usepackage[dvipsnames]{xcolor}
\def\SBcomment[#1]{\textcolor{Red}{#1}}
\def\SWcomment[#1]{\textcolor{Green}{#1}}
\def\SScomment[#1]{\textcolor{Blue}{#1}}

\begin{document}

\maketitle
\noindent\SWcomment[Like in the dafx paper, ]\SBcomment[you can make comments like this!] \SScomment[Also you Stefania :)]
\section{Continuous-time}\label{sec:continuous}
The PDE for the bowed stiff string looks like:
\begin{equation}
    u_{tt} = c^2u_{xx}-\kappa^2u_{xxxx}-2\sigma_0u_t+2\sigma_1u_{txx}-J(x_\text{B})f(v,z)/\rho A
\end{equation}
with wave speed $c = \sqrt{T/\rho A}$ and stiffness $\kappa = \sqrt{EI/\rho A}$ and force function
\begin{equation}
    f(v, z) = s_0z + s_1\dot z + s_2v + s_3w
\end{equation}
with relative velocity $v$ and average bristle displacement $z$. The time derivative of $z$ -- the rate of change of the bristle displacement -- is defined as \SWcomment[like this right?]{}
\begin{equation}\label{eq:zdot}
    \dot z = r(v,z) = v\bigg[1-\alpha(v, z)\frac{z}{z_\text{ss}(v)}\bigg]
\end{equation}
\SBcomment[yes...]

\SBcomment[One thing...can't you just substitute (3) into (2) and get rid of the $\dot{z}$ variable entirely?]\SWcomment[ and then just use this single equation for vector NR? Don't you need 2 equations to solve for 2 variables?]{}
\SBcomment[No, just momentarily confused...] \SWcomment[Because I think that in the code, we do substitute, or at least, calculate $\dot{z}$ as described in (3) and then use this value in (2). But I suppose you figured that :)]{}
\SBcomment[OK, I think it would help a lot, in framing things, to rewrite the above as]
\begin{equation}
    \dot{z} = r(v,z)
\end{equation}
\SBcomment[So that you don't have to carry around $\dot{z}$ as a variable anymore...]
with adhesion map
\begin{equation}\label{eq:alpha}
\alpha(v, z) 
    \begin{cases}
    \begin{rcases}
        &0 & |z| \leq z_\text{ba}\\
       &\alpha_\text{m}(v,z)&\quad \  z_\text{ba}<|z|<|z_\text{ss}(v)|\\
        &1 &|z|\geq |z_\text{ss}(v)|
        \end{rcases}\text{if}\  \sgn(v)=\sgn(z)\\
        0\qquad \qquad \qquad \qquad \qquad \qquad \quad \, \text{if}\  \sgn(v)\neq\sgn(z),
    \end{cases}
\end{equation}
the transition between the elastic and plastic behaviour
\begin{equation}
    \alpha_\text{m} = \frac{1}{2}\big[1+\sin\big(\sgn(z)\Phi\big)\big]\quad \text{where}\quad \Phi = \pi\frac{z-\sgn(z)\frac{1}{2}(|z_\text{ss}(v)|+z_\text{ba})}{|z_\text{ss}(v)|-z_\text{ba}},
\end{equation}
and the steady-state function
\begin{equation}\label{eq:zss}
    z_\text{ss} = \frac{\sgn(v)}{s_0}\Big[f_\text{C}+(f_\text{S}-f_\text{C})e^{-(v/v_\text{S})^2}\Big].
\end{equation}
\section{Discrete-time}
The FDS for the bowed string looks like:
\begin{equation}\label{eq:FDS1}
    \delta_{tt} u_l^n =c^2 \delta_{xx} u_l^n -\kappa^2\delta_{xxxx} u_l^n - 2\sigma_0\delta_{t\cdot} u_l^n
    + 2\sigma_1\delta_{t-}\delta_{xx}u_l^n - J(x_\text{B}^n)f(v^n, z^n)/\rho A,
\end{equation}
where $x_\text{B} \in [0,\hdots, L]$. If we use a truncating function rather than an interpolator and using the following notation
\begin{equation}
    I_0(x_\text{B}^n)u^n_l \Rightarrow u^n_{x_\text{B}}
\end{equation}
the following description for the relative velocity at the bowing point $x_\text{B}$ can be defined
\begin{equation}
    v^n = \delta_{t\cdot}u^n_{x_\text{B}} - v_\text{B}^n \quad \Rightarrow \quad \delta_{t\cdot}u^n_{x_\text{B}} = v^n + v_\text{B}^n.
\end{equation}
Then, equation \eqref{eq:FDS1} can be rewritten at the bowing point $l=x_\text{B}$:
\begin{equation}
\label{eq:FDS}
\delta_{tt} u_{x_\text{B}}^n =c^2 \delta_{xx} u_{x_\text{B}}^n -\kappa^2\delta_{xxxx} u_{x_\text{B}}^n - 2\sigma_0(v^n + v_\text{B}^n)
+ 2\sigma_1\delta_{t-}\delta_{xx}u_{x_\text{B}}^n - f(v^n, z^n)/\rho Ah,
\end{equation}
where we need to iteratively solve for two unknown variables: the relative velocity between the bow and the string ($v$) and the mean bristle displacement ($z$).
The excitation (or bowing) function is defined as
\begin{equation}
    f(v^n, z^n) = s_0z^n + s_1 r^n + s_2v^n + s_3w^n\ .
\end{equation}
We can use $\alpha^n = \alpha(v^n, z^n)$ and $z_\text{ss}^n = z_\text{ss}(v^n)$ as the discrete counterparts of \eqref{eq:alpha} and \eqref{eq:zss} respectively, to define the discrete counterpart of $r$ in \eqref{eq:zdot}
\begin{equation}
    r^n = r(v^n,z^n)= v^n\bigg[1-\alpha^n\frac{z^n}{z_\text{ss}^n}\bigg]\ .
\end{equation}
We can rewrite \eqref{eq:FDS} using the following identities
\begin{equation}
    \delta_{tt}u_{x_\text{B}}^n = \frac{2}{k}\big(\delta_{t\cdot}u_{x_\text{B}}^n-\delta_{t-}u_{x_\text{B}}^n\big) \quad \text{ and } \quad \delta_{t\cdot}u_{x_\text{B}}^n = v^n + v_\text{B}^n,
\end{equation}

\noindent resulting in 

\begin{equation}
\label{eq:stiffStringFDS}
\frac{2}{k}v^n + \frac{2}{k}v_\text{B}^n - \frac{2}{k}\delta_{t-}u_{x_\text{B}}^n =c^2 \delta_{xx} u_l^n -\kappa^2\delta_{xxxx} u_{x_\text{B}}^n - 2\sigma_0v^n - 2\sigma_0v_\text{B}^n
+ 2\sigma_1\delta_{t-}\delta_{xx}u_{x_\text{B}}^n - f(v^n, z^n)/\rho Ah,
\end{equation}
\\
which can be rewritten to
\begin{equation}\label{eq:newtonFunction}
\boxed{g_1(v^n, z^n) = \Big(\frac{2}{k} + 2\sigma_0\Big)v^n + \frac{s_0z^n+s_1r^n+s_2v^n+s_3w^n}{\rho A h} + b^n = 0} \end{equation}
where
\begin{equation}
    b^n = \frac{2}{k}v_\text{B}^n-\frac{2}{k}\delta_{t-}u_{x_\text{B}}^n - c^2 \delta_{xx} u_{x_\text{B}}^n +\kappa^2\delta_{xxxx} u_{x_\text{B}}^n + 2\sigma_0v_\text{B}^n
- 2\sigma_1\delta_{t-}\delta_{xx}u_{x_\text{B}}^n
\end{equation}
can be pre-computed. Using $a^n$ in \eqref{eq:an}, a second function can be defined
\begin{equation}
   \boxed{g_2(v^n,z^n) = r^n -  a^n = 0}
\end{equation}
where
\begin{equation}\label{eq:an}
    a^n = (\mu_{t-})^{-1}\delta_{t-}z^n = \frac{2}{k}(z^n-z^{n-1})-a^{n-1}
\end{equation} \SWcomment[Changed the forward to backward operators. Still a bilinear transform right?]{}
\SBcomment[What happens if you just use (11) as written instead of this...] \SWcomment[you mean solving (11) substituted in (10) for $\dot{z}$?]{}\SBcomment[Hey, no, I take that back, this looks ok actually]
describes the bilinear transform of $z$. 
This function, together with \eqref{eq:newtonFunction} can be used for multivariate newton raphson:

\begin{equation}
    \begin{bmatrix}
    v_{(i+1)}^n\\
    z_{(i+1)}^n
    \end{bmatrix}
    =
    \begin{bmatrix}
    v_{(i)}^n\\
    z_{(i)}^n
    \end{bmatrix}
    -
    \begin{bmatrix}
    \frac{\partial g_1}{\partial v} & \frac{\partial g_1}{\partial z}\\
    \frac{\partial g_2}{\partial v} & \frac{\partial g_2}{\partial z}\\
    \end{bmatrix}^{-1}
    \begin{bmatrix}
    g_1\\
    g_2
    \end{bmatrix}
\end{equation}
where the derivatives are defined as follows (see Section \ref{sec:derivations} for derivations of some of these): 
\begin{align}
    \frac{\partial g_1}{\partial v} &= \frac{2}{k} + 2\sigma_0 + \frac{s_1}{\rho Ah}\frac{\partial r}{\partial v}+\frac{s_2}{\rho Ah}\\
    \frac{\partial g_1}{\partial z} &= \frac{s_0}{\rho A h} + \frac{s_1}{\rho Ah}\frac{\partial r}{\partial z}\\
    \frac{\partial g_2}{\partial v} &= \frac{\partial r}{\partial v}
    \\\frac{\partial g_2}{\partial z}&= \frac{\partial r}{\partial z} -\frac{2}{k}
\end{align}
with
\begin{align}
    \frac{\partial r}{\partial v} &= 1-z^n\Bigg(\frac{(\alpha^n+\frac{\partial \alpha^n}{\partial v}v^n)z_\text{ss}^n - \frac{\partial z_\text{ss}^n}{\partial v}\alpha^n v^n}{(z_\text{ss}^n)^2}\Bigg)\label{eq:g2v}\\
    \frac{\partial r}{\partial z} &= -\frac{v^n}{z_\text{ss}^n}\bigg(\frac{\partial \alpha^n}{\partial z}z^n + \alpha^n\bigg)
    \end{align}
    and
    \begin{align}
    \frac{\partial\alpha^n}{\partial v} &=\sgn(z_\text{ss}^n)\frac{\partial z_\text{ss}^n}{\partial v}\frac{z_\text{ba} - |z^n|}{(|z_\text{ss}^n| - z_\text{ba})^2}\frac{\pi}{2}\cos\big(\sgn(z^n)\Phi\big)\\
    \frac{\partial\alpha^n}{\partial z}&=\frac{\sgn(z^n)\pi\cos\big(\sgn(z^n)\Phi\big)}{2(|z_\text{ss}^n|-z_\text{ba})}\\
    \frac{\partial z_\text{ss}^n}{\partial v} &= -\frac{2|v^n|}{v_\text{S}^2 s_0}(f_\text{S}-f_\text{C})e^{-(v^n/v_\text{S})^2}
\end{align}

\section{Order of calculation}
Essentially, the order of calculation happens from bottom to top in section \ref{sec:continuous} after which the derivatives are calculated in this same order. We calculate the error term ($\epsilon$) right after the equations are calculated using the values for $v$ and $z$ at the $i$'th iteration. If we calculate the error term at the end of the loop (between steps 11. and 12.) we will have obtained values for $v_{(i+1)}$ and $z_{(i+1)}$, but not yet for $z_\text{ss}(v_{(i+1)})$, $\alpha(v_{(i+1)}$, $z_{(i+1)})$ and $r(v_{(i+1)}, z_{(i+1)})$. 
\begin{algorithm}[h]\label{alg:calcOrder}
\fbox{\parbox{0.93\linewidth}{
 \For{t = 1:lengthSound}{
 calculate computable part $b^n$\\
 $i = 0$\\
  \While{$\epsilon <$ tol}{
    calculate..\\
    1. $z_\text{ss}(v_{(i)}^n)$\\
    2. $\alpha(v_{(i)}^n,z_{(i)}^n)$\\
    3. $r(v^n_{(i)}, z^n_{(i)})$\\
    4. $g_1$, $g_2$\\
    5. Calculate $\epsilon$: $\epsilon = \Big|r(v^n_{(i)}, z^n_{(i)}) - r(v^n_{(i-1)}, z^n_{(i-1)})\Big|$\\
    6.--10. Compute derivatives of 1.--4. in the same order. \\
    11. Perform vector NR to obtain $v_{(i+1)}$ and $z_{(i+1)}$\\
    12. Increment $i$: $i = i + 1$
  }
  Calculate $\mathbf{u}^{n+1}$ using the values for $v^n$ and $z^n$ from the NR. 
 }
 }}
 \caption{Pseudocode showing the current order of calculation.}
\end{algorithm}

% \section{Notes from before}
% Newton's method (or Newton-Raphson) is defined as
% \begin{equation}
%     x^{i+1} = x^{i} - \frac{g(x^i)}{g'(x^i)}
% \end{equation}
% where $g(x)$ is an arbitrary function dependent on to-be-calculated variable $x$.
% As we need to find the roots of \eqref{eq:zdot}, $g$ with $x=\dot z$ is defined as
% \begin{equation}
%   g(\dot z) = v\bigg[1-\alpha(v, z)\frac{z}{z_\text{ss}(v)}\bigg] -  \dot z(v,z) = 0,
% \end{equation}
% and
% \begin{equation}
%     \dot z^{i+1} = \dot z^i - \frac{g(\dot z^i)}{g'(\dot z^i)}
% \end{equation}
% where
% \begin{equation}
%     g'(\dot z^i) = -0.2295\frac{dg(\dot z^i)}{dv} + \frac{k}{2}\frac{dg(\dot z^i)}{dz} - 1.
% \end{equation}
% In the iteration, we use the newly calculated value for $\dot z$ and the value of $z$ at the previous time step to calculate an estimate of $z$ using the trapezoidal rule
% \begin{equation}
%     z^n = z^{n-1} + \frac{k}{2}(\dot z^{i+1} + \dot z^{n-1}).
% \end{equation}
% Inserting this into \eqref{eq:newtonFunction} we can caluclate $v$ using
% \begin{equation}
%     v = \frac{-s_0z-s_1\dot z-b}{s_2 + \frac{2}{k} + 2\sigma_0}.
% \end{equation}
\appendix
\section{Derivations}
\label{sec:derivations}
\subsubsection*{Derivative of $r$ with respect to $v$}
\begin{equation}
    \begin{aligned}
        \frac{\partial r}{\partial v} &= \frac{\partial}{\partial v} \Bigg(v^n\bigg[1-\alpha^n\frac{z^n}{z_\text{ss}^n}\bigg]\Bigg)\\
        &= 1 - \frac{\partial}{\partial v}\Bigg(\frac{\alpha^nv^nz^n}{z_\text{ss}^n}\Bigg)\\
        &\\
        &\begin{gathered}
        \text{Quotient rule: }\boxed{\Bigg(\frac{f}{g}\Bigg)' = \frac{f'g - fg'}{g^2}}
        \end{gathered}
        \\
        &\\
        &= 1 - \frac{\frac{\partial}{\partial v}(\alpha^n v^n z^n)z_\text{ss}^n - \frac{\partial z_\text{ss}^n}{\partial v}\alpha^n v^n z^n}{(z_\text{ss}^n)^2}
        \\
        &\\
        &\begin{gathered}
        \text{Product rule: }\boxed{\big(f\cdot g\big)' = f'g + fg'}
        \end{gathered}
        \\
        &\\
        &= 1-\Bigg(\frac{(\alpha^nz^n + \frac{\partial \alpha^n}{\partial v}v^nz^n)z_\text{ss}^n - \frac{\partial z_\text{ss}^n}{\partial v}\alpha^n v^nz^n}{(z_\text{ss}^n)^2}\Bigg)\\
        &= 1-z^n\Bigg(\frac{(\alpha^n+\frac{\partial \alpha^n}{\partial v}v^n)z_\text{ss}^n - \frac{\partial z_\text{ss}^n}{\partial v}\alpha^n v^n}{(z_\text{ss}^n)^2}\Bigg)
    \end{aligned}
\end{equation}
\subsubsection*{Derivative of $r$ with respect to $z$}
\begin{equation}
    \begin{aligned}
        \frac{\partial r}{\partial z} &= \frac{\partial}{\partial z} \Bigg(v^n\bigg[1-\alpha^n\frac{z^n}{z_\text{ss}^n}\bigg]\Bigg)\\
        &= \frac{\partial}{\partial z}\Bigg(-\frac{\alpha^nv^nz^n}{z_\text{ss}^n}\Bigg)
        \\
        &\\
        &\begin{gathered}
        \text{Product rule: }\boxed{\big(f\cdot g\big)' = f'g + fg'}
        \end{gathered}
        \\
        &\\
        & = -\frac{v^n}{z_\text{ss}^n}\bigg(\frac{\partial \alpha^n}{\partial z}z^n + \alpha^n\bigg)
    \end{aligned}
\end{equation}
\subsubsection*{Derivative of $\alpha$ with respect to $v$: $\frac{\partial \alpha^n}{\partial v}$}
\begin{equation}
    \begin{aligned}
        \frac{\partial \alpha^n}{\partial v} &= \frac{\partial}{\partial v}\Bigg(\frac{1}{2}\big[1+\sin(\sgn(z^n)\Phi)\big]\Bigg)\\
        &= \frac{\partial}{\partial v}\Big(\sgn(z^n)\Phi\Big)\frac{1}{2}\cos\big(\sgn(z^n)\Phi\big)\\
        &= \frac{\partial}{\partial v}\Bigg(\frac{\sgn(z^n)\pi\big(z^n-\sgn(z^n)\frac{1}{2}(|z_\text{ss}^n|+z_\text{ba})\big)}{|z_\text{ss}^n|-z_\text{ba}}\Bigg)\frac{1}{2}\cos\big(\sgn(z^n)\Phi\big)\\
        &\\
        &\begin{gathered}
        \text{Quotient rule: }\boxed{\Bigg(\frac{f}{g}\Bigg)' = \frac{f'g - fg'}{g^2}}
        \end{gathered}
        \\
        &\\
        &= \frac{-\frac{\pi}{2}\frac{\partial |z_\text{ss}^n|}{\partial v}(|z_\text{ss}^n|-z_\text{ba}) - \frac{\partial |z_\text{ss}^n|}{\partial v}\Big(\sgn(z^n)\pi\big[z^n-\sgn(z^n)\frac{1}{2}(|z_\text{ss}^n|+z_\text{ba})\Big]\bigg)}{(|z_\text{ss}^n| - z_\text{ba})^2}\frac{1}{2}\cos\big(\sgn(z^n)\Phi\big)\\
        &= \frac{\partial |z_\text{ss}^n|}{\partial v}\frac{-\frac{\pi}{2}(|z_\text{ss}^n|-z_\text{ba}) + \frac{\pi}{2}(|z_\text{ss}^n|+z_\text{ba})-\sgn(z^n)\pi z^n}{(|z_\text{ss}^n| - z_\text{ba})^2}\frac{1}{2}\cos\big(\sgn(z^n)\Phi\big)\\
        &= \frac{\partial |z_\text{ss}^n|}{\partial v}\frac{z_\text{ba} - |z^n|}{(|z_\text{ss}^n| - z_\text{ba})^2}\frac{\pi}{2}\cos\big(\sgn(z^n)\Phi\big)
        &\\
        &\begin{gathered}
        \text{Derivative of absolute value: }\boxed{|f|' = \frac{f}{|f|}f'= \sgn(f)f'}
        \end{gathered}
        \\
        &\\
        &=\sgn(z_\text{ss})\frac{\partial z_\text{ss}^n}{\partial v}\frac{z_\text{ba} - |z^n|}{(|z_\text{ss}^n| - z_\text{ba})^2}\frac{\pi}{2}\cos\big(\sgn(z^n)\Phi\big)
    \end{aligned}
\end{equation}
\subsubsection*{Derivative of $\alpha$ with respect to $z$: $\frac{\partial \alpha^n}{\partial z}$}
\begin{equation}
    \begin{aligned}
    \frac{\partial \alpha^n}{\partial z} &= \frac{\partial}{\partial z} \Bigg(\frac{1}{2}\big[1+\sin(\sgn(z^n)\Phi)\big]\Bigg)\\
    &= \frac{\partial}{\partial z}\Big(\sgn(z^n)\Phi\Big)\frac{1}{2}\cos\big(\sgn(z^n)\Phi\big)\\
    &= \frac{\partial}{\partial z}\Bigg(\frac{\sgn(z^n)\pi\big(z^n-\sgn(z^n)\frac{1}{2}(|z_\text{ss}^n|+z_\text{ba})\big)}{|z_\text{ss}^n|-z_\text{ba}}\Bigg)\frac{1}{2}\cos\big(\sgn(z^n)\Phi\big)\\
    &= \frac{\sgn(z^n)\pi}{|z_\text{ss}^n|-z_\text{ba}}\frac{1}{2}\cos\big(\sgn(z^n)\Phi\big)\\
    &= \frac{\sgn(z^n)\pi\cos\big(\sgn(z^n)\Phi\big)}{2(|z_\text{ss}^n|-z_\text{ba})}
    \end{aligned}
\end{equation}
\subsubsection*{Derivative of $z_\text{ss}$ with respect to $v$}
\begin{equation}
\begin{aligned}
    \frac{\partial z_\text{ss}^n}{\partial v} &= \frac{\partial}{\partial v} \Bigg(\frac{\sgn(v^n)}{s_0}\Big(f_\text{C}+(f_\text{S}-f_\text{C})e^{-(v^n/v_\text{S})^2}\Big)\Bigg)\\
    &= \frac{\partial}{\partial v}\big(-(v^n/v_\text{S})^2\big)\frac{\sgn(v^n)}{s_0}(f_\text{S}-f_\text{C})e^{-(v^n/v_\text{S})^2}\\
    &= \frac{-2|v^n|}{v_\text{S}^2s_0}(f_\text{S}-f_\text{C})e^{-(v/v_\text{S})^2}
    \end{aligned}
\end{equation}

\end{document}
